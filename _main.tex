\documentclass[]{book}
\usepackage{lmodern}
\usepackage{amssymb,amsmath}
\usepackage{ifxetex,ifluatex}
\usepackage{fixltx2e} % provides \textsubscript
\ifnum 0\ifxetex 1\fi\ifluatex 1\fi=0 % if pdftex
  \usepackage[T1]{fontenc}
  \usepackage[utf8]{inputenc}
\else % if luatex or xelatex
  \ifxetex
    \usepackage{mathspec}
  \else
    \usepackage{fontspec}
  \fi
  \defaultfontfeatures{Ligatures=TeX,Scale=MatchLowercase}
\fi
% use upquote if available, for straight quotes in verbatim environments
\IfFileExists{upquote.sty}{\usepackage{upquote}}{}
% use microtype if available
\IfFileExists{microtype.sty}{%
\usepackage{microtype}
\UseMicrotypeSet[protrusion]{basicmath} % disable protrusion for tt fonts
}{}
\usepackage[margin=1in]{geometry}
\usepackage{hyperref}
\hypersetup{unicode=true,
            pdftitle={실무 빅데이터 분석을 위한 SQL과 R의 활용 CAMP},
            pdfauthor={ChanYub Park},
            pdfborder={0 0 0},
            breaklinks=true}
\urlstyle{same}  % don't use monospace font for urls
\usepackage{longtable,booktabs}
\usepackage{graphicx,grffile}
\makeatletter
\def\maxwidth{\ifdim\Gin@nat@width>\linewidth\linewidth\else\Gin@nat@width\fi}
\def\maxheight{\ifdim\Gin@nat@height>\textheight\textheight\else\Gin@nat@height\fi}
\makeatother
% Scale images if necessary, so that they will not overflow the page
% margins by default, and it is still possible to overwrite the defaults
% using explicit options in \includegraphics[width, height, ...]{}
\setkeys{Gin}{width=\maxwidth,height=\maxheight,keepaspectratio}
\IfFileExists{parskip.sty}{%
\usepackage{parskip}
}{% else
\setlength{\parindent}{0pt}
\setlength{\parskip}{6pt plus 2pt minus 1pt}
}
\setlength{\emergencystretch}{3em}  % prevent overfull lines
\providecommand{\tightlist}{%
  \setlength{\itemsep}{0pt}\setlength{\parskip}{0pt}}
\setcounter{secnumdepth}{5}
% Redefines (sub)paragraphs to behave more like sections
\ifx\paragraph\undefined\else
\let\oldparagraph\paragraph
\renewcommand{\paragraph}[1]{\oldparagraph{#1}\mbox{}}
\fi
\ifx\subparagraph\undefined\else
\let\oldsubparagraph\subparagraph
\renewcommand{\subparagraph}[1]{\oldsubparagraph{#1}\mbox{}}
\fi

%%% Use protect on footnotes to avoid problems with footnotes in titles
\let\rmarkdownfootnote\footnote%
\def\footnote{\protect\rmarkdownfootnote}

%%% Change title format to be more compact
\usepackage{titling}

% Create subtitle command for use in maketitle
\newcommand{\subtitle}[1]{
  \posttitle{
    \begin{center}\large#1\end{center}
    }
}

\setlength{\droptitle}{-2em}
  \title{실무 빅데이터 분석을 위한 SQL과 R의 활용 CAMP}
  \pretitle{\vspace{\droptitle}\centering\huge}
  \posttitle{\par}
  \author{ChanYub Park}
  \preauthor{\centering\large\emph}
  \postauthor{\par}
  \predate{\centering\large\emph}
  \postdate{\par}
  \date{2017-04-06}


\usepackage{amsthm}
\newtheorem{theorem}{Theorem}[chapter]
\newtheorem{lemma}{Lemma}[chapter]
\theoremstyle{definition}
\newtheorem{definition}{Definition}[chapter]
\newtheorem{corollary}{Corollary}[chapter]
\newtheorem{proposition}{Proposition}[chapter]
\theoremstyle{definition}
\newtheorem{example}{Example}[chapter]
\theoremstyle{remark}
\newtheorem*{remark}{Remark}
\begin{document}
\maketitle

{
\setcounter{tocdepth}{1}
\tableofcontents
}
\chapter*{머리말}
\addcontentsline{toc}{chapter}{머리말}

이 책은 \href{http://www.fastcampus.co.kr}{패스트캠퍼스}의
\href{http://www.fastcampus.co.kr/category_data_camp/}{데이터 사이언스
캠프} 코스의 \href{http://www.fastcampus.co.kr/data_camp_dabrp/}{실무
빅데이터 분석을 위한 SQL과 R의 활용 CAMP}의 수업용으로 제작되었습니다.
수업에 대해 더 자세히 알고 싶으신 분은
\href{http://www.fastcampus.co.kr/data_camp_dabrp_instructor_1/}{강사
인터뷰}를 참고하세요.

책은 각각 R의 IDE로 사실상 표준인
\href{http://www.rstudio.org/}{RStudio}에 사용하기 좋은 기능 소개,
대부분의 에러 문제를 해결할 수 있는 기초 자료형에 대한 이해, 단순 반복
업무를 위한 for문과 apply류 맛보기, 데이터 원본/의존성의 개념과 SQL 문법
익히기, \href{http://vita.had.co.nz/papers/tidy-data.pdf}{tidy data}
개념과 dplyr+tidyr로 데이터 다루기, 보고용 차트를 위한 ggplot2 사용하기,
정기 보고서 자동 작성을 위해 knitr로 문서화하고 스케줄러로 자동화하기,
\href{https://shiny.rstudio.com/}{shiny} 패키지를 활용한 인터렉티프 웹
만들기로 구성되어 있습니다. 많은 부분
\href{https://www.rstudio.com/wp-content/uploads/2016/01/rstudio-IDE-cheatsheet.pdf}{rstudio-IDE-cheatsheet},
\href{http://r4ds.had.co.nz/}{R for Data Science},
\href{https://bookdown.org/rdpeng/rprogdatascience/}{R Programming for
Data Science}, \href{http://had.co.nz/ggplot2/}{ggplot2-book},
\href{https://shiny.rstudio.com/tutorial/}{shiny-tutorial},
\href{https://courses.edx.org/courses/course-v1:Microsoft+DAT204x+1T2017/info}{Microsoft's
DAT204x},
\href{https://www.datacamp.com/community/tutorials/}{datacamp},
\href{https://www.programiz.com/r-programming/}{programiz}을
참고하였습니다.

데이터 분석은 많은 단계들과 업무들로 나누어져 있습니다. 개인적으로 1.
데이터 확보 2. 데이터 정제 3. 분석 4. 시각화의 단계를 거친다고
생각합니다. 데이터란 사내에서 관리하고 있는 내부 데이터와 외부 인터넷에
공개되어 있는 데이터로 구분할 수 있습니다. 이런 데이터들 중 분석과 업무
목적에 맞는 데이터를 찾고, 활용하기 위에 확보하는 과정이 1단계 입니다.
분석 방법과 내용에 따라 확보된 데이터를 정리하거나 고쳐야 하는 일도
있습니다. 2 단계는 그것을 뜻합니다. 3 단계인 분석은 다양한 통계적
방법들을 통해 분석 목적을 이루는 것입니다. 4단계는 이렇게 이루어낸
결과물을 다른 사람에게 전달하기 위해 필요합니다.

이 책은 그 중 2단계인 정제와 4단계인 시각화에 초점이 맞춰져 있습니다.
1단계중 외부 데이터 확보를 위한 웹크롤링은 준비중입니다.

저작물 라이선스로
\href{https://creativecommons.org/licenses/by-nc-nd/4.0/}{크리에이티브
커먼즈 라이선스 4.0(저작자 표시-비영리-변경 금지(BY-NC-ND))}를 따릅니다.

\chapter*{정보}
\addcontentsline{toc}{chapter}{정보}

이 책의 소스는
\href{https://github.com/mrchypark/data_camp_dabrp}{여기}에서 확인하실
수 있으며 아래 주어진 R 패키지(및 종속 패키지)의 버전으로
제작되었습니다. 저작물의 재현을 위해서 필요합니다.

\begin{longtable}{lllll}
\toprule
package & * & version & date & source\\
\midrule
backports &  & 1.0.5 & 2017-01-18 & CRAN (R 3.3.2)\\
bookdown &  & 0.3.14 & 2017-03-22 & Github (rstudio/bookdown@f427fdf)\\
devtools &  & 1.12.0 & 2016-06-24 & CRAN (R 3.3.3)\\
digest &  & 0.6.12 & 2017-01-27 & CRAN (R 3.3.3)\\
evaluate &  & 0.10 & 2016-10-11 & CRAN (R 3.3.3)\\
\addlinespace
htmltools &  & 0.3.5 & 2016-03-21 & CRAN (R 3.3.3)\\
knitr &  & 1.15.1 & 2016-11-22 & CRAN (R 3.3.3)\\
magrittr &  & 1.5 & 2014-11-22 & CRAN (R 3.3.3)\\
memoise &  & 1.0.0 & 2016-01-29 & CRAN (R 3.3.3)\\
packrat &  & 0.4.8-1 & 2016-09-07 & CRAN (R 3.3.3)\\
\addlinespace
Rcpp &  & 0.12.9 & 2017-01-14 & CRAN (R 3.3.3)\\
rmarkdown &  & 1.3 & 2016-12-21 & CRAN (R 3.3.3)\\
rprojroot &  & 1.2 & 2017-01-16 & CRAN (R 3.3.3)\\
stringi &  & 1.1.2 & 2016-10-01 & CRAN (R 3.3.3)\\
stringr &  & 1.2.0 & 2017-02-18 & CRAN (R 3.3.3)\\
\addlinespace
withr &  & 1.0.2 & 2016-06-20 & CRAN (R 3.3.3)\\
yaml &  & 2.1.14 & 2016-11-12 & CRAN (R 3.3.3)\\
\bottomrule
\end{longtable}

이 책은 by mrchypark on 목요일, 3월 23, 2017 오전 1:52:18 KST에
마지막으로 업데이트 했습니다.

\chapter{시작하기 전에}\label{before-start}

\section{준비된 데이터}\label{data-for-class}

\chapter{RStudio에 사용하기 좋은 기능 소개}\label{rstudio}

\section{RStudio 소개}\label{introduce-rstudio}

\subsection{IDE 란}\label{ide}

\subsection{다른 IDE 소개}\label{ide-others}

\subsection{Rstudio}\label{rstudio}

\section{구조와 기능들}\label{structures}

\subsection{source 창}\label{source-pane}

\subsection{console 창}\label{console-pane}

\subsection{Environment 창}\label{env-pane}

\subsection{help 창}\label{help-pane}

\section{프로젝트와 버전관리}\label{project-git}

\subsection{폴더와 프로젝트}\label{use-project}

\subsection{버전관리도구 git}\label{introduce-git}

\subsection{오픈소스와 github}\label{introduce-github}

\section{도움말}\label{help-page}

\subsection{도움말 설명}\label{introduce-help}

\subsection{도움말 사용법}\label{help-usage}

\subsection{함수의 옵션 설명}\label{help-param}

\subsection{함수의 상세 설명}\label{help-detail}

\subsection{예시 코드}\label{help-example}

\chapter{대부분의 에러 문제를 해결할 수 있는 기초 자료형에 대한
이해}\label{data-type}

\section{기초 자료형}\label{basic-data-type}

\section{벡터}\label{vector}

\section{메트릭스}\label{metrix}

\section{팩터}\label{factor}

\section{리스트}\label{list}

\section{데이터프레임}\label{data-frame}

\section{날짜}\label{date-time}

\chapter{단순 반복 업무를 위한 for문과 apply류 맛보기}\label{for-apply}

\section{반복문}\label{introduce-for}

\subsection{for}\label{for}

\subsection{while}\label{while}

\subsection{repeat}\label{repeat}

\subsection{break}\label{break}

\subsection{next}\label{next}

\section{조건문}\label{introduce-if}

\subsection{if}\label{if}

\subsection{else}\label{else}

\subsection{ifelse}\label{ifelse}

\subsection{switch}\label{switch}

\section{apply류의 함수들}\label{introduce-apply}

\subsection{apply}\label{apply}

\subsection{*apply \{\#applys\}}\label{apply-applys}

\subsection{aggregate}\label{aggregate}

\chapter{데이터 원본/의존성의 개념과 SQL 문법 익히기}\label{sql}

\section{사전 준비}\label{prerequisite}

\subsection{mariaDB 설치}\label{install-mariadb}

\subsubsection{windows}\label{mariadb-windows}

\subsubsection{mac}\label{mariadb-mac}

\subsubsection{docker}\label{mariadb-docker}

\section{데이터를 바라보는 시선}\label{for-data}

\subsection{데이터 원본}\label{original-data}

\subsection{데이터 의존성}\label{data-dependency}

\section{기본 SQL 문법}\label{basic-sql}

\subsection{CRUD와 데이터베이스}\label{crud}

\subsection{Read 문법}\label{read}

\subsection{Join}\label{join}

\chapter{tidy data 개념과 dplyr+tidyr로 데이터 다루기}\label{tidyr}

\section{tidy data}\label{tidy-data}

\section{tidyverse}\label{tidyverse}

\chapter{보고용 차트를 위한 ggplot2 사용하기}\label{ggplot}

\section{test}\label{test}

\chapter{정기 보고서 자동 작성을 위해 knitr로 문서화하고 스케줄러로
자동화하기}\label{knitr}

\section{Markdown 문법}\label{markdown}

\section{Rmd로 문서 작성}\label{rmd}

\section{스케줄러}\label{scheduler}

\chapter{shiny 패키지를 활용한 인터렉티프 웹 만들기}\label{shiny}

\chapter{마치며}\label{outro}


\end{document}
